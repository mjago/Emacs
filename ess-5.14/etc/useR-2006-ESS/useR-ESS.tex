\documentclass{beamer}

\mode<presentation>
{
  \usetheme{Warsaw}
  \setbeamercovered{transparent}
}


\usepackage[english]{babel}
\usepackage[latin1]{inputenc}
\usepackage{times}
\usepackage[T1]{fontenc}
% Or whatever. Note that the encoding and the font should match. If T1
% does not look nice, try deleting the line with the fontenc.


\title[Emacs and R]{Emacs enhances data analysis and programming with R}
\subtitle{ESS useR Short Course} 
\author[Rossini]{A.J.~Rossini}

\institute[Novartis Pharma AG]
{
  Modeling and Simulation \\
  Novartis Pharma AG \\
  Basel, Switzerland
}

\date[14.06.2006]
{14.06.2006 / useR-2006}

\subject{Talks}
% This is only inserted into the PDF information catalog. Can be left
% out. 

% Delete this, if you do not want the table of contents to pop up at
% the beginning of each subsection:
% \AtBeginSubsection[]
% {
%   \begin{frame}<beamer>
%     \frametitle{Outline}
%     \tableofcontents[currentsection,currentsubsection]
%   \end{frame}
% }


\begin{document}

\begin{frame}
  \titlepage
\end{frame}

\section{Introduction}

\begin{frame}
  \frametitle{Outline}
  %% \tableofcontents[hideallsubsections]
  \begin{enumerate}
  \item Introduction (now, 15 minutes)
  \item Using Emacs (45 minutes)
  \item Using ESS (60 minutes)
  \item Exercise 1: ESS
  \item Exercise 2: Sweave
  \item Emacs extensions (30 minutes)
  \item Emacs Lisp (30 minutes)
  \item Discussion and Misc Topics (related Emacs tools, ESS extensions, future designs)
  \end{enumerate}
\end{frame}

\subsection[Goals]{Goals for Today}

\begin{frame}
  \frametitle{\#1: Efficient Program Editing}
  \framesubtitle{Edit, and let the computer repeat}
  \begin{itemize}
  \item Weird archaic keystrokes are dominant
  \item Weird archaic keystrokes can make life easier
  \item Menus and toolbars help Emacs
  \item Knowing R and data analysis means we can enhance related activities.
  \item Folding editors, Object explorers
  \end{itemize}
\end{frame}

\begin{frame}
  \frametitle{\#2: Process Interaction}
  \framesubtitle{You enter text for the process, so edit\ldots}
  \begin{itemize}
  \item R vs. ``emacs -f R''
  \item Getting Help
  \item Approaches for interacting
  \item Practical Directory layout
  \end{itemize}
\end{frame}

\begin{frame}
  \frametitle{\#3: Emacs is Extended and Extensible}
  \framesubtitle{Lisp and other hackers are your friends}
  \begin{itemize}
  \item Introduction to Lisp and Emacs Lisp
  \item Lisp extends Emacs
  \item Design of ESS
  \item Future Extensions??
  \end{itemize}
\end{frame}

%% \begin{frame}
%%   \frametitle{Make Titles Informative.}
%%   You can create overlays\dots
%%   \begin{itemize}
%%   \item using the \texttt{pause} command:
%%     \begin{itemize}
%%     \item
%%       First item.
%%       \pause
%%     \item    
%%       Second item.
%%     \end{itemize}
%%   \item
%%     using overlay specifications:
%%     \begin{itemize}
%%     \item<3->
%%       First item.
%%     \item<4->
%%       Second item.
%%     \end{itemize}
%%   \item
%%     using the general \texttt{uncover} command:
%%     \begin{itemize}
%%       \uncover<5->{\item
%%         First item.}
%%       \uncover<6->{\item
%%         Second item.}
%%     \end{itemize}
%%   \end{itemize}
%% \end{frame}

\begin{frame}
  \frametitle{Disclaimer}
  \begin{itemize}
  \item ESS (and Emacs) is many (different) things to different
    people.
  \item Even for R development, the variation in personal philosophies
    and approaches for use is extremely high (even in the Core
    development team).
  \item Religious issues focusing on Emacs should be discussed elsewhere 
    (c.f. \url{http://www.dina.kvl.dk/~abraham/religion/})
  \end{itemize}
\end{frame}

\begin{frame} \frametitle{Demo!}
  \framesubtitle{It'll get boring before it gets better\ldots}
  \alert{Please note or comment on any part that you'd like covered later.}
\end{frame}

\subsection{Introducing Emacs}

\begin{frame} \frametitle{Overview}
  \begin{itemize}
  \item Emacs is the extensible, customizable, self-documenting
    real-time display editor.
  \item One of the oldest and yet still most powerful display editors
  \item The name \textbf{Emacs} comes from Editor (or Extensible)
    MACroS. (source of other amusing acronym expansions)
  \item Originally written in 1976 as a set of extensions to TECO
    (Text Editor and COrrector).  It has since evolved.
  \end{itemize}
\end{frame}

\begin{frame} \frametitle{Pros and Cons for Emacs}
  \begin{itemize}
  \item \alert{Why use Emacs?}  It is a powerful text editor which
    can be extended to be a general interface to a computer (everything can be done within it).
  \item It is highly portable (in its own way) across many platforms.
  \item \alert{Why do not use it?} It has a different user interface,
    which is understandable given its age (little change since 1985).
  \item It does not follow ``modern'' keybindings.
  \item People don't understand Lisp (otherwise, what they want to do
    could be done in Emacs)
  \end{itemize}
  \centerline{\alert{some people are allergic to it}}
\end{frame}

\begin{frame}
  \frametitle{On-line Resources}
  Links to links:
  \begin{itemize}
  \item Emacs Wiki: \url{http://www.emacswiki.org/}
  \item Emacs \url{http://www.gnu.org/software/emacs/}
  \item XEmacs \url{http://www.xemacs.org/}
  \item Emacs Lisp Library \url{http://www.damtp.cam.ac.uk/user/sje30/emacs/ell.html}
  \end{itemize}
\end{frame}

\begin{frame}
  \frametitle{The Emacsen Problem}
  \framesubtitle{different religions cause stress}
  There are many variants or dialects of Emacs (\textit{similar to the
    S language having AT\&T's S, Insightful's S-PLUS, and R-core's R})
  \begin{itemize}
  \item Emacs: ``classic version''
  \item XEmacs: ``experiments'' (GUIs, dynamic loading, packaging)
  \item SXEmacs: ``splinter'' (Unix-focus, streamlining XEmacs)
  \end{itemize}
  
  ESS support is best in order (Emacs, XEmacs, SXEmacs).  Most active
  ESS developers use Emacs.  Users seem to be split 50/50 between
  Emacs and XEmacs.\\
  No bug reports on SXEmacs\ldots 
\end{frame}

\begin{frame}
  \frametitle{Installation}
  \framesubtitle{the bane of Emacs}
  Installation can be tricky:
  \begin{itemize}
  \item Linux: Emacs usually can be installed or is preinstalled
  \item Microsoft: XEmacs is easier to install, Emacs might be better
    supported.
  \item Mac OSX: Emacs, AquaEmacs; via Fink, or UCLA package, or\ldots
    (check on the list).
  \end{itemize}
  We aren't dealing with this here!
\end{frame}

\begin{frame} \frametitle{Editing methodology}
  \framesubtitle{Emacs resembles modern editors unlike dual-mode VI}

  \begin{itemize}
  \item Idea: every keystroke (or sequence of keystrokes) generates a
    list to evaluate (i.e. command, or lisp function with arguments).
  \item Pressing ``X'' evaluates the list (self-insert-command "X")  (we'll talk about Lisp later)    
  \item Modifiers extend the keyboard (CTL, SHIFT, ALT, pressed
    simultaneously, some such as ESC, META, and others are pressed
    first, released, and then followed by a key).
  \end{itemize}
\end{frame}

\subsection{Introducing ESS}

\begin{frame}
  \frametitle{Overview}
  \framesubtitle{ESS augments Emacs for R}
  \begin{itemize}
  \item Emacs Speaks Statistics (or Emacs Statistical System)
  \item Oldest active statistically-focused OSS project (1989).
  \item Design goal: a consistent interface unifying batch and
    interactive data analysis styles.
  \item Novelty (at the time): process control, unified interface. 
  \item String (stream) operations; ``cheap parser''.
  \item Supports R, S-PLUS, SAS, Stata, LispStat
  \item Current development team (n=6) spans the US and Europe.
  \end{itemize}
\end{frame}

\begin{frame} \frametitle{History}
  \begin{itemize}
  \item 3 branches: S-mode, SAS-mode, Stata-mode
  \item S-mode: 1989, 2 groups managing the project before (Bates/F
    Ritter/E Kademan, M Meyer, DM Smith (now Insightful)).
    R added in 1995 (Rossini/Maechler)
  \item SAS: '94, Tom Cook ('92, John Sall, SAS).  Integrated '95-6, Rossini
  \item Stata-mode: 1997, Thomas Lumley. Added 1997 (Rossini). 
  \item 1997: last major restructuring (``grand unification'')
  \item 2004: switch leaders: Maechler takes over
  \end{itemize}
  \alert{We are still suffering from the 1997 grand unification.}
\end{frame}

\begin{frame}
  \frametitle{Interactive Programming}
  \framesubtitle{Everything goes back to being Lisp}
  \begin{itemize}
  \item Interactive programming (as originating with Lisp): works
    extremely well for data analysis (Lisp being the original
    ``programming with data'' language).
  \item Theories/methods for how to do this are reflected in styles
    for using ESS.
  \end{itemize}
  \alert{Good Statistical Analysis is on-line Interactive Programming}
\end{frame}

\begin{frame}
  \frametitle{Installation}
  \framesubtitle{We have to deal with this here, though\ldots}
  \begin{itemize}
  \item Fetch the ESS package (tar or zip-file)
  \item unpack
  \item add a line in \textbf{.emacs} to load the \textbf{ess-site.el} file.
  \item restart Emacs (or \textbf{M-x load-file /path/to/ess-site.el}).
  \end{itemize}
\end{frame}

\begin{frame}
  \frametitle{Review: Introduction}
  \begin{itemize}
  \item ESS is a program in Lisp to extend Emacs
  \item Characteristics: OSS, developed initially to make life easier;
    continued development supported research in enhanced data analysis
    interfaces.
  \item Emacs has strengths and weaknesses; these are usually
    different than percieved strengths and weaknesses.
  \end{itemize}
\end{frame}

\section{Using Emacs}

\subsection{Overview: Emacs}

\begin{frame} \frametitle{Goals}
  \begin{itemize}
  \item use of Emacs to edit files
  \item filesystem interaction
  \end{itemize}
\end{frame}

\subsection{Getting Started}

\begin{frame}[fragile]
  \frametitle{Starting (X)Emacs}
  \framesubtitle{trivial but essential}
  \alert{emacs} can be used instead of \alert{xemacs}.
\begin{verbatim}
xemacs <filename>  run xemacs, edit <filename>
xemacs -nw         run xemacs in _text_ mode
                   (nw="No Window"; useful for 
                    slow remote connections)
\end{verbatim}
  \begin{itemize}
  \item Once started, no need for more.
  \item \alert{Only run one Emacs session}.  This is a cardinal rule,
    do not violate it.
    (c.f.\hyperlink{keystrokes:fileloading}{\beamerbutton{file-loading section}})
  \end{itemize}
\end{frame}


\begin{frame}[fragile]
  \frametitle{Breaking and Quitting!}
  \framesubtitle{essential knowledge}
  \label{emacs:basics}
  \begin{itemize}
  \item Decide you were wrong while Emacs works: \verb+C-g+  (universal break)
  \item Quit and terminate: \verb+C-x C-c+ (will usually prompt you
    for any files which are modified but not saved)
  \end{itemize}
\end{frame}


\begin{frame}[fragile]
  \frametitle{Understanding Keystrokes.}
  \begin{itemize}
  \item \verb+[tab]+ is the TAB (indent) key.
  \item \verb+[return]+ is the Return (carriage return, enter) key.
  \item \textbf{C-h} means press \alert{control key} AND ``\alert{h}'' key at same time.
  \item \textbf{ESC-h} means press \alert{ESC key} THEN ``\alert{h}''
  \item \textbf{M-h} means press \alert{ALT key} AND ``\alert{h}''
  \item \textbf{M-C-h} means press Meta or Alt while pressing h.  OR
    (if no meta/alt): ESC THEN (control and h keys together).
  \end{itemize}
  Older keyboards (and sometimes older Macs) without ALT or Meta lead
  to confusion between ESC and Meta, but ideally they should be different.
\end{frame}

\begin{frame} \frametitle{Stopping Activity}
  \framesubtitle{you need to know this\ldots}
  \alert{C-g} stops the current Emacs activity to the best of its ability.
\end{frame}

\begin{frame}
  \frametitle{Windows and Frames}
  \begin{itemize}
  \item Window:  a different view on (usually) a different buffer within the same application
  \item Frame: a different window/application in the windowing system.
  \end{itemize}
\end{frame}

\begin{frame}[fragile]
  \frametitle{File suffices drive mode setup}
  \framesubtitle{Use the right file name formats:}
  \begin{itemize}
  \item \verb+.S+, such as \verb+critical_simulation.S+
  \item \verb+.R+, such as \verb+speedy_simulation.R+
  \end{itemize}
  There are other ways to force a file to be edited in a particular
  mode, but at this point, keep it simple.
\end{frame}

\begin{frame}
  \frametitle{Commands}
  
  Emacs has a single command-entry mode.  There are 2 approaches for commands:
  \begin{itemize}
    \item \textbf{M-x (command-name)}
    \item keystrokes/toolbars/menus (which implement the above)
  \end{itemize}
  
\end{frame}

\subsection{Keystrokes}

\begin{frame}
  \frametitle{Learning the Strokes.}
  \begin{itemize}
  \item forced practice: painful but fast.  Keep a copy of the
    reference cards nearby!  (usually 2-3 days of severe pain,
    followed by bliss).
  \item accelerators and menus can help: but they quickly become a crutch.
  \item Consider working through the tutorial (\textbf{C-h t}).
  \end{itemize}
  \alert{this is a major issue for speed typists converting to/from
    Emacs usage.}
\end{frame}

\begin{frame}[fragile]
  \frametitle{Loading Files}
  \label{keystrokes:fileloading}
  \begin{itemize}
  \item \verb+C-x C-f+    (load new file, \textbf{not} deleting old)
  \item \verb+C-x 4 C-f+  (load new file in different window)
  \item \verb+C-x 5 C-f+  (load new file in different frame)
  \end{itemize}
  (c.f. \hyperlink{emacs:basics}{\beamerbutton{basics}})
\end{frame}

\begin{frame}[fragile]
  \frametitle{Changing Active Buffer (edit file)}
  To edit (operate on) another file:
  \begin{itemize}
  \item \verb+C-x b+    (switch to new buffer in same window)
  \item \verb+C-x 4 b+  (replaces current file in different window)
  \item \verb+C-x 5 b+  (replaces current file in different frame)
  \end{itemize}
  Buffers represent the files you are editing (explain).
\end{frame}

\begin{frame}
  \frametitle{Review of Common Keystrokes}
  \framesubtitle{Demo and pain}
\end{frame}


\subsection{In Any Emacs Buffer}

\begin{frame}[fragile] \frametitle{Moving Around}
\begin{verbatim}
C-v    Move forward one screenful
M-v    Move backward one screenful
C-l    Clear and redraw screen
M- ->  Meta-<right> - moves forward a word 
M- <-  Meta-<left> - moves back a word
M- |^  Meta-<up> - move up a paragraph
M- V   Meta-<down> - move down a paragraph
M- <   Meta-<less than> - move to file start
M- >   Meta-<greater than> - move to file end
\end{verbatim}
\end{frame}

\begin{frame}[fragile] \frametitle{Cut and Paste}
\begin{verbatim}
C-d    _D_elete
C-k    _K_ill from the cursor position to
       end of line
C-y    Recover/Paste (_Y_ank) killed text 
       (repeat to copy) 
M-y    recover former killed text (after C-y).
       Repeat to go back through stack).
C-x u  _U_ndo
\end{verbatim}
\end{frame}


\begin{frame}[fragile] \frametitle{Loading / Saving Files}
\begin{verbatim}
C-x C-f  _F_ind a file
C-x C-s  _S_ave the file

If you find a second file with C-x C-f, 
the first file remains inside Emacs.  
You can switch back to it by finding it 
again with C-x C-b.  This way you can get 
quite a number of files inside Emacs.
 \end{verbatim}
\end{frame}


\begin{frame}[fragile] \frametitle{Managing Buffers and Windows}
\begin{verbatim}
C-x 0      Move between windows  
           (Oh, not Zero!)
C-x 1      One window 
           (i.e., kill all other windows).
C-x b      Switch to new _b_uffer
C-x C-b    List _b_uffers
\end{verbatim}
\end{frame}


\begin{frame}[fragile] \frametitle{Search and Replace}
\begin{verbatim}
M-x (then) replace-string
           Replace string
M-x (then) query-replace-string
           Will ask you, for each match,
           if you really want to replace
           the old string with the new one.
C-s        _S_earch forward (repeat to
           reuse past search strings)
C-r        Search _R_everse (repeat to
           reuse past search strings)
\end{verbatim}
\end{frame}

                     
\begin{frame}[fragile] \frametitle{Misc Emacs Actions}
\begin{verbatim}
C-h or C-h ?    _H_elp
C-h c (command) _H_elp on this _c_ommand

C-u 8 (character or command)   
             Repeat character/command 8 times

C-g          Stop, unhang.
C-x C-c      Stop and exit (_c_lose) Emacs  
\end{verbatim}
\end{frame}


\begin{frame}
  \frametitle{Point and Mark}
  \framesubtitle{critical concepts to learn}

  \begin{itemize}
  \item Point:  where the cursor is.
  \item Mark: a location that you've decided (actively or inactively)
    is important.
  \item Example:   Move cursor somewhere.  \textbf{M-x set-mark} or
    \textbf{C-(space)}.
  \item Exchange point and mark:   \textbf{M-x
      exchange-point-and-mark} or \textbf{C-x C-x}.
  \end{itemize}
\end{frame}

\subsection{Review: Using Emacs}

\begin{frame}
  \frametitle{Review}
  \framesubtitle{Emacs}
  \begin{itemize}
  \item \textbf{C-g} stops emacs activity and returns to edit mode; \textbf{C-x C-c} quits
  \item Try to use the keyboard and not the mouse, menus, or toolbars.
  \item Keypresses are simply commands.
  \item Biggest user headache?  Differenting commands and text entry
  \item Buffers represent the files you are editing.  Morally, they
    are the files themselves (with important exceptions).
  \end{itemize}
\end{frame}


\section{Using ESS}

\subsection{Overview: ESS}

\begin{frame}
  \frametitle{Overview: ESS}
  This section should provide an overview for 
  \begin{itemize}
  \item editing
  \item interaction with R
  \item facilitating help
  \item reusing transcripts
  \item Sweave support
  \end{itemize}
  and finally, \alert{putting it all together}
\end{frame}

\subsection{Editing}

\begin{frame}[fragile]
  \frametitle{Coding Style}
  \framesubtitle{Indentation makes code readable}

  Comment conventions:
  \begin{enumerate}
  \item \verb+#+ gets moved to the right-side of the line
  \item \verb+##+ is placed at the current indentation level
  \item \verb+###+ is placed flush-left.
  \end{enumerate}
\end{frame}

\begin{frame}[fragile]
  \frametitle{Indentation Example}
\begin{verbatim}
    my.x <- rnorm(10)
### This is flush
    ## but this is at level
                         # this is right
\end{verbatim}
\end{frame}


\begin{frame}[fragile]
  \frametitle{Editing, ESS adds\ldots}
  \begin{itemize}
  \item Automatic fontification, indentation
  \end{itemize}
  If R is running in another buffer:
  \begin{itemize}
  \item evaluation of lines (C-c C-n), regions (C-c C-r), functions
    (C-c C-f), and buffers (C-c C-b)
  \item help output in other buffers (and via command, C-c C-v)
  \item Object (data, function) completion
    \begin{itemize}
    \item \verb+C-c [tab]+ 
    \item \verb+[tab]+ (sometimes).
    \end{itemize}
  \item Filename completion: M-\[tab\]
  \end{itemize}
\end{frame}

\begin{frame}[fragile]
  \frametitle{R editing buffer}
  \framesubtitle{Usually named for the file being edited}
  (mode line says "ESS[S]")
\begin{verbatim}
C-c C-j        Send line to R
C-c C-n        Send line to R 
               and go to _n_ext line
C-c C-r        Send highlighted _r_egion to R
C-c C-b        Send whole _b_uffer to R
C-c C-f        Send _f_unction surrounding
               cursor to R
C-c C-v xxx    get R help on "xxx"
\end{verbatim}
\end{frame}

\subsection{Interactive}

\begin{frame}
  \frametitle{Run R}
  \begin{itemize}
  \item Within emacs
  \item At the command line
  \end{itemize}
\end{frame}

\begin{frame}[fragile]
  \frametitle{R process within Emacs (on start)}
  \framesubtitle{starting from the command line or launcher}
  Replace the command \textbf{R} with one of:
\begin{verbatim}
emacs -f R        run (x)emacs, starting R
xemacs -f R       session                   
\end{verbatim}
  adds a few nice features to the command interface.  (follow with
  \textbf{\&} to background the process; \verb+xemacs+ might also
  work)
\end{frame}

\begin{frame}[fragile]
  \frametitle{R process within Emacs (running emacs)}
  \framesubtitle{Starting from a running emacs}
\begin{verbatim}
M-x R         Start R process buffer
M-x R-mode    Change mode of current buffer
              for R code editing
\end{verbatim}
\end{frame}

\begin{frame}[fragile]
  \frametitle{Evaluate file contents}
  \framesubtitle{working from the editing (ESS) buffer}. 

  From a file of commands:
  \begin{itemize}
  \item Send current line to R: \verb+C-c C-j+
  \item Send current function to R (assumes the cursor is in the
    function body): \verb+C-c C-f+.
  \item Send current region (highlighted) to R  \verb+C-c C-r+
  \item Send whole buffer/file to R  \verb+C-c C-b+
  \end{itemize}
\end{frame}

\begin{frame}[fragile]
  \frametitle{Object and File completion}
  Completion of objects (functions and data):
  \begin{enumerate}
  \item \verb+C-c [tab]+
  \item \verb+[tab]+ (sometimes).
  \end{enumerate}
\end{frame}

\begin{frame}
  \frametitle{Dumping Objects}
  \framesubtitle{Getting objects back}
  (for modification, saving, eventual reloading):
  \begin{enumerate}
  \item \textbf{C-c C-d} dumps the named object into an editing buffer
  \item Examples: today's date for use as a log file; (140606), ``lm''.
  \item if object doesn't exist: create a blank buffer
  \item if object exists: insert the object's textual representation (no comments)
  \item The file will be stored in \textbf{/tmp/username.objectname.R}
    (unless you want it elsewhere or named something else).
  \end{enumerate}
  Type commands into the edit file, evaluate them when desired, edit
  as needed for correctness (and comment mistakes, with reasons for
  them, for later reminders and documentation).
\end{frame}

\begin{frame}[fragile]
  \frametitle{Commandline History}
  \framesubtitle{ESS remembers what you've done}
  Search command-line history ``manually'', matching on current input
  \begin{enumerate}
  \item backwards: \verb+M-p+
  \item forwards:  \verb+M-n+
  \end{enumerate}
  Complete current line based on command-line history:
  \begin{enumerate}
  \item backwards: \verb+\C-[uparrow]-p+
  \item forwards:  \verb+M-n+
  \end{enumerate}
  If you are reviewing old commands, to re-enter it (i.e. cursor on that line)
  \verb+[return]+.  
\end{frame}

\begin{frame}
  \frametitle{Philosophy: Files are ``truth''}
  \framesubtitle{Don't believe your .RData}
  \begin{itemize}
  \item This is a generally decent approach.
  \item Saving objects in binary format in special files can be a
    reasonable strategy
  \item Work from an R file, \emph{submitting} commands to the R
    process.  This saves the source, and allows for structuring of the
    input, along with having a transcript of the order of commands to
    understand reproducibility aspects.
  \end{itemize}
\end{frame}

\begin{frame}[fragile]
  \frametitle{Working from R process}
  \framesubtitle{from the process (inferior ESS, or iESS) buffer}
  \alert{not usually recommended}
  \begin{enumerate}
  \item One improvement is to use a \verb+source()+-like facility.  To
    source a file into R (within ESS): \textbf{C-c C-l} (load file)
  \item To find what the errors are (if any): \textbf{C-c `} (backquote).
  \end{enumerate}
  The latter will tell you what the error is.  You'll have to go back
  to the file (\textbf{C-x C-b} or use \textit{Buffers} menu item to
  bring in back to the screen) and correct it.
\end{frame}

\begin{frame}[fragile]
  \frametitle{Working from R buffer}
  \framesubtitle{from the editing (ESS) buffer}
  Better yet, use a file, and send material from the file straight to
  the process.  This works if the file ends in the proper suffix (and
  hence, the mode at the bottom should say ``ESS'').  The following is
  possible:
  \begin{itemize}
  \item Send current line to R: \textbf{C-c C-j}
  \item Send current function to R (assumes the cursor is in the
    function body): \textbf{C-c C-f}
  \item Send current region (highlighted) to R: \textbf{C-c C-r}
  \item Send whole buffer/file to R: \textbf{C-c C-b}
  \end{itemize}
\end{frame}

\begin{frame}
  \frametitle{Philosophy: RData is true}
  \framesubtitle{Editing objects without source}
  \begin{itemize}
  \item This is a dangerous approach; corruption of the RData file is
    more difficult to fix using an editor or other external tools than
    corruption of text files, usually.
  \item Trusting .RData to have what you've done is rather suboptimal
    strategy (quite prone to user mistakes)
  \end{itemize}
\end{frame}

\begin{frame}
  \frametitle{If you must use a command line\ldots}
  \begin{itemize}
  \item This style is useful for quick and dirty exercises/testing.
  \item Transcripts can be cleaned and edited to record the salient
    points of the sesssion.
  \end{itemize}
\end{frame}

\subsection{Putting it together}

\begin{frame}
  \frametitle{Begin ESS/Emacs/R}
  \begin{enumerate}
  \item Start XEmacs  (in a terminal window, enter \textbf{xemacs \&})
  \item Start R (\textbf{M-x R}).
  \item Load a file (\textbf{C-x C-f} \textit{filename})
  \end{enumerate}
\end{frame}

\begin{frame}[fragile]
  \frametitle{Data Analysis Example}
  Demo.
\end{frame}

\begin{frame}[fragile] \frametitle{Function Editing}
  \begin{enumerate}
  \item demo.R: evaluate.
  \item Try editing existing functions: \textbf{my.test.function} and
    \textbf{that}, by \textbf{C-c C-d this}.  Remember that you should
    be able to complete the name using \textit{ajr.t}\textbf{SPACE}
    (i.e.  Control-C, then SPACE key).  Generally, completion of
    variables and functions can be done using \textbf{C-c [tab]} in the
    editing buffer, and \textbf{SPACE} in the command entry area (at
    the very bottom of the Emacs screen/window).
  \end{enumerate}
\end{frame}

\subsection{Special Facilities and Activties}

\begin{frame}
  \frametitle{Navigating Help Buffers}
  \begin{itemize}
  \item To get help on a function (help(``lm'')) without
    spoiling your input line:  \textbf{C-c C-v}
  \item Navigation of sections: ``n'' next section, ``p'' prev section.
  \item evaluation of example: ``l'' (eval line) ``r'' eval region.
  \item searching: ``/'' isearch forward
  \item kill-buffer and go \textit{back}: ``x''
  \end{itemize}
  See menu items
\end{frame}

\begin{frame}
  \frametitle{Using Transcripts}
  \framesubtitle{for teaching, reproduction, and audits}
  \begin{itemize}
  \item Transcript suffix:  .Rt
  \item Stepping through entries:
  \item Cleaning transcripts: \textbf{M-x ess-transcript-clean-buffer}
  \item Cleaning transcripts: \textbf{M-x ess-transcript-DO-clean-region} (even when read-only toggled)
  \end{itemize}
\end{frame}

\begin{frame}
  \frametitle{Using Multiple Processes}
  \framesubtitle{For when you want to multitask}
  ESS supports multiple processes.   This is useful for 
  \begin{itemize}
  \item testing on multiple versions of R
  \item testing on R and S-PLUS (and S4, if one still has a copy)
  \item inefficient multiprocessing
  \end{itemize}
\end{frame}

\begin{frame}
  \frametitle{Using Remote Processes}
  \framesubtitle{loading other people's computers}
  Since we are capturing/using R text I/O, local or remote is not an
  issue (running R in an Emacs shell, logged in (shell account via
  telnet, ssh, rlogin) to another machine)
  \begin{itemize}
  \item \textbf{M-x ESS-remote}
  \item follow instructions
  \end{itemize}
\end{frame}

\begin{frame}[fragile]
  \frametitle{Customizing ESS}
  See (commented out) example customizations in ess-site.el (assuming
  that the lisp files are installed:
\begin{verbatim}
M-x locate-library [return] ess-site [return]
\end{verbatim}
  will help.  Also, 
\begin{verbatim}
M-x customize-group [return] ess
\end{verbatim}
  helps to review possible customizations (this is an alternative to
  editing the \textbf{\~{}/.emacs} file for setting customizing
  variables).
\end{frame}

\subsection{Sweave Support}

\begin{frame}
  \frametitle{Overview: Sweave}
  Sweave originated from the Literate Data Analysis philosophy which
  in itself was the idea of applying Literate Programming to the
  specialized programming discipline \emph{interactive data analysis}.

  Support for editing Noweb files existed within ESS 4 years before
  Sweave, but Sweave is a ``Killer App''.
\end{frame}

\begin{frame}
  \frametitle{Documents as Programs}
  \begin{itemize}
  \item Every electronic document is a ``computer program'' (sometimes
    at a primitive level) for some rendering device or program (such
    as: Microsoft Word; Postscript printers; Adobe Acrobat Reader;
    Emacs)
  \item This is the general theory behind the WWW, as well as the
    so-called \emph{Semantic Web}.
  \end{itemize}
\end{frame}

\begin{frame}
  \frametitle{Literate Programming} 
  A programming methodology devised by Donald Knuth in the late 70s,
  early 80s, which mixes computer programs doing mathematics with
  mathematical documentation (formulas, theorems, algorithms).

  2 operations:
  \begin{itemize}
  \item Tangling: translating document into undocumented computer source code for
    compiling.
  \item Weaving: translating document into documentation for the source code.
  \end{itemize}
  The original tool was called Web (predates the \textit{World-Wide-Web}).
\end{frame}

\begin{frame} \frametitle{Literate Statistical Practice}
  \framesubtitle{or \emph{Literate Data Analysis:}}
  The application of literate programming to statistics; recognizing
  that statistical data analysis is a specialized programming activity
  with different characteristics than systems programming or
  applications programming.

  (realization prototype: ESS-Noweb and Sweave).
\end{frame}

\begin{frame} \frametitle{ESS enhances Sweave}
  \begin{itemize}
  \item context specific modes (doc: latex, html, muse; code: R, etc).
  \item chunk and thread evaluation.
  \item functions for processing Sweave files.
  \end{itemize}
\end{frame}

\begin{frame}
  \frametitle{Evaluation}
  \framesubtitle{Chunks and Threads}
  \begin{itemize}
  \item eval-chunk (run code inside a code chunk)
  \item eval-thread (run code inside a series of code chunks)
  \end{itemize}
\end{frame}

\begin{frame}
  \frametitle{Accelerated Sweave Development }
  \begin{itemize}
  \item ESS has specialized support
  \item use Makefiles (editor-neutral approach)
  \item \textbf{M-x ess-makeSweave}    or \textbf{M-n s}
  \item \textbf{M-x ess-makeLatex}     or \textbf{M-n l}
  \item \textbf{M-x ess-makePS}        or \textbf{M-n p}  
  \item \textbf{M-x ess-makePDF}       or \textbf{M-n P}
  \item \textbf{M-x ess-insert-Sexpr}  or \textbf{M-n x}
  \end{itemize}
\end{frame}

\subsection{Review: ESS}

\begin{frame}
  \frametitle{Review: Using ESS}
  \begin{itemize}
  \item ESS for R has 4 modes: editing, interactive, transcript, help.
  \item Focus on learning to use keystrokes rather than mouse.
  \end{itemize}
\end{frame}

\section{Emacs is Extended}

\begin{frame}
  \frametitle{Packages and Extensions.}
  Emacs has many modes: 
  \begin{itemize}
  \item language-specific markup (programming, markup)
  \item editor behavior (spell checking, pending delete, folding,
    editor emulation (vi, brief))
  \item IDE capabilities (object explorers, completers)
  \item database (edb)
  \item communication (mail, news, www, rdf)
  \item sub (inferior) process control (telnet, R, \ldots)
  \item publication (text mode conversion)
  \item version control interfaces
  \item anything you can program!    
  \end{itemize}
\end{frame}

\begin{frame}
  \frametitle{Relevant to Data analysis and R}
  \begin{itemize}
  \item folding modes (outline; however, the newer version, allout,
    stomps on R-mode)
  \item object explorers (ecb)
  \item documentation modes
    \begin{itemize}
    \item specialized:  latex, html, rtf
    \item generic: muse, \ldots (idea: simple markup which translates to
      html, latex, texinfo, etc).
    \end{itemize}
  \end{itemize}
\end{frame}

\begin{frame} 
  \frametitle{Dired}
  A directory/file explorer and navigator.
\end{frame}

\begin{frame}
  \frametitle{Emacs Code Browser}
  \begin{itemize}
  \item ESS integrates to support R
  \item Extension/customization: through imenu support.
  \item file-level navigation and objects, editing history of files,
    files in current directory, directory tree.
  \end{itemize}
  (demo with R-examples/FCS.R)
\end{frame}

\begin{frame}
  \frametitle{Muse}
  \framesubtitle{a mode for documentation and notes}
  \begin{itemize}
  \item Wiki-like editing, hyperlinks, within text.
  \item simple format
  \item Supports generation of latex, pdf, (x)html, xml.
  \end{itemize}
\end{frame}

\begin{frame}
  \frametitle{Planner: Task management}
  \framesubtitle{getting organized and tracking activity}
  \begin{itemize}
  \item Plain-text task management and organization
  \item Supports many common time-management approaches (GTD, ).
  \item Extremely customizeable
  \end{itemize}
\end{frame}

\subsection{LaTeX Extensions}

\begin{frame}
  \frametitle{LaTeX Support}
  \framesubtitle{nearly WYSIWYG}
  \begin{itemize}
  \item AUC-TeX: latex document support
  \item BiBTeX-mode and RefTeX: citation management
  \item X-Symbol: native display of mathematical symbols
  \item tex-preview: You can see what you will get (YCSWYWG)
  \end{itemize}
\end{frame}

\section{Lisp and Emacs Lisp}

\begin{frame}
  \frametitle{Customizing Emacs}
  \framesubtitle{Sometimes with Lisp}
  \begin{itemize}
  \item Critical components for customizing Emacs: current best
    practices suggest that you customize via ``M-x customize''. (this
    is a recent (last 5 years) change!).
  \item However, adding Lisp commands to your \textbf{.emacs} file is 
    still very useful (and sometimes required).
  \item Note that commas denote comments (from the comma until the end
    of the line).
  \end{itemize}
\end{frame}

\subsection{Introduction to Lisp}

\begin{frame}[fragile]
  \frametitle{Lisp}
  Lisp (LISt Processor) is different than most high-level computing
  languages, and similar to Emacs, it is very old (1956).  Lisp is
  built on lists of things which might be evaluated:
\begin{verbatim}
(function data1 data2 data3)
\end{verbatim}
  or ``quoted'':
\begin{verbatim}
'(function data1 data2 data3)
\end{verbatim}
  The difference is important -- lists of data (the second) are not
  functions applied to data (the first).
\end{frame}

\begin{frame}[fragile]
  \frametitle{Evaluation}
  Lisp is built on lists of things which might be evaluated:
\begin{verbatim}
(function data1 data2 data3)
\end{verbatim}
  or ``quoted'':
\begin{verbatim}
'(function data1 data2 data3)
(list function data1 data2 data3)
\end{verbatim}
  The difference is important -- lists of data (the second) are not
  functions applied to data (the first).
\end{frame}

\begin{frame}[fragile]
  \frametitle{Defining Variables}
  \framesubtitle{Setting variables}
\begin{verbatim}
(setq <variable> <value>)
\end{verbatim}
  Example:
\begin{verbatim}
(setq ess-source-directory
      "/home/rossini/R-src")
\end{verbatim}
\end{frame}

\begin{frame}[fragile]
  \frametitle{Defining on the fly}
\begin{verbatim}
(setq ess-source-directory
   (lambda () (file-name-as-directory
         (expand-file-name
           (concat (default-directory)
                   ess-suffix "-src")))))
\end{verbatim}
  (Lambda-expressions are anonymous functions, i.e. ``instant-functions'') 
\end{frame}


\begin{frame}[fragile]
  \frametitle{Function Reuse}
  By naming the function, we could make the previous example reusable
  (if possible):
\begin{verbatim}
(defun my-src-directory ()
      (file-name-as-directory
         (expand-file-name
           (concat (default-directory)
                   ess-suffix "-src"))))
\end{verbatim}
  Example:
\begin{verbatim}
(setq ess-source-directory (my-src-directory))
\end{verbatim}
\end{frame}

\begin{frame}[fragile]
  \frametitle{Programming Emacs Lisp} 

  There are three general approaches to testing, evaluating, and
  deploying Emacs Lisp code.
  \begin{itemize}
  \item Loading files:  \verb+M-x load-file <filename.el>+
  \item Using the \verb+*scratch*+ buffer
  \item \verb+ielm+ , an emacs-lisp REPL
  \end{itemize}
  Review the latter two
\end{frame}

\begin{frame}[fragile]
  \frametitle{Playing in the scratch buffer.}
  \begin{enumerate}
  \item \verb+C-x b *scratch*+
  \item (display-time)\verb+C-j+
  \item (setq my-test ``asdf'')\verb+C-j+
  \item my-test\verb+C-j+
  \end{enumerate}
\end{frame}

\begin{frame}
  \frametitle{ielm: command-line usage.}
  \begin{enumerate}
  \item M-x ielm
  \item follow the instructions for more information (similar to using
    R in interactive mode)
  \end{enumerate}
\end{frame}

\subsection{Extending Emacs and ESS}

\begin{frame}
  \frametitle{Organization.}
  \framesubtitle{Structure of ESS}
  \begin{itemize}
  \item How ESS works
  \item Reading the code
  \item making extensions
  \end{itemize}
\end{frame}

\begin{frame}
  \frametitle{Debugging.}
  \begin{itemize}
  \item Set options to ``debug on error'' and ``debug on signal''.
    Both bring up the stack-trace of the current status of the lisp
    program (nesting and variable values).
  \item Look for errors in loading to determine the path taken to the error
  \item more sophisticated debuggers exist.
  \end{itemize}
\end{frame}


\begin{frame}
  \frametitle{Idea generation: SLIME}
  \framesubtitle{Lisp IDE suggests ESS improvements}
  \begin{itemize}
  \item Use R subprocesses (old idea of DTL).
  \item Synchronize processes
  \end{itemize}
\end{frame}


\section{Discussion}

\begin{frame}
  \frametitle{Future?}
  \begin{itemize}
  \item Continued support for R within ESS
  \item Object and formating support
  \end{itemize}
\end{frame}


\begin{frame}
  \frametitle<presentation>{Summary}

  % Keep the summary *very short*.
  \begin{itemize}
  \item
    Emacs takes time to learn, but rewards the user.
  \item
    ESS enhances R.
  \item
    Lisp is worth learning.
  \end{itemize}
  
  % The following outlook is optional.
  \vskip0pt plus.5fill
  \begin{itemize}
  \item
    Outlook
    \begin{itemize}
    \item
      ESS is the oldest open-source statistics project; yet it can
      always be improved!
    \item
      Many possible extensions for supporting work modes for
      Statisticians and Data Analysts.  More need implementation.
    \end{itemize}
  \end{itemize}
\end{frame}

\end{document}
